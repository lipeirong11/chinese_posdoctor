\chapter{作者简历及攻读学位期间发表的学术论文与研究成果}

%\textbf{本科生无需此部分}。

\section*{作者简历}

\subsection*{李培荣}

\section*{已发表(或正式接受)的学术论文(博士期间):}

%[1] ucasthesis: A LaTeX Thesis Template for the University of Chinese Academy of Sciences, 2014.
%\begin{itemize}
\begin{enumerate}
\item ``Measurements of absolute hadronic branching fractions of $\Lambda_{c}^{+}$ baryon'', [BESIII Collaboration], Phys.\ Rev.\ Lett.\  {\bf 116} 052001 (2016)

\item ``Observation of a neutral charmoniumlike state $Z_c(4025)^0$ in $e^{+} e^{-} \to (D^{*} \bar{D}^{*})^{0} \pi^0$'', [BESIII Collaboration], Phys.\ Rev.\ Lett.\  {\bf 115}, no. 18, 182002 (2015)

\item ``Observation of a charged charmoniumlike structure in $e^+e^- \to (D^{*} \bar{D}^{*})^{\pm} \pi^\mp$ at $\sqrt{s}=4.26$GeV'', [BESIII Collaboration], Phys.\ Rev.\ Lett.\  {\bf 112}, no. 13, 132001 (2014)

\item ``Recent charmed baryon results at BESIII'', Pei-Rong Li, Nuclear and Particle Physics Proceedings, Volumes 270-272, January-March 2016, Pages 123-126

%\end{itemize}
\end{enumerate}

\section*{已发表(或正式接受)的学术论文(博士后期间):}

\begin{enumerate}
\item ``Measurements of Absolute Branching Fractions for $\Lambda^+_c\to\Xi^0K^+$ and $\Xi(1530)^0K^+$'', [BESIII Collaboration], arXiv.1803.04299(2018)
\item ``Measurement of $\psitwos$ production cross-sections in proton-proton collisions at $\sqs=13\tev$'', [LHCb Collaboration], LHCb-ANA-2016-023(2018)
\end{enumerate}


%\section*{申请或已获得的专利:}
%
%(无专利时此项不必列出)

%\section*{参加的研究项目及获奖情况:}
%
%可以随意添加新的条目或是结构。

\chapter[致谢]{致\quad 谢}\chaptermark{致\quad 谢}% syntax: \chapter[目录]{标题}\chaptermark{页眉}
\thispagestyle{noheaderstyle}% 如果需要移除当前页的页眉
%\pagestyle{noheaderstyle}% 如果需要移除整章的页眉

感谢博士后合作导师郑阳恒教授对我提供的良好宽松的科研环境。

感谢博士后科学基金会和中国科学院联合资助基金的资助。

\cleardoublepage[plain]% 让文档总是结束于偶数页,可根据需要设定页眉页脚样式,如 [noheaderstyle]
