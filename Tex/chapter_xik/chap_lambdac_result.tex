\section{$\LtoXiK$和$\LtoXisK$分支比结果展示}
将我们用上述章节获得的各个测量值,依据公式~\ref{eq:br}可以直接计算两个衰变道的分支比$\mathcal{B}$。
我们得到$\mathcal{B}(\LtoXiK)=(5.90\pm0.86\pm0.39)\times10^{-3}$和$\mathcal{B}(\LtoXisK)=(5.02\pm0.99\pm0.31)\times10^{-3}$,其中第一项误差为统计误差,第二项为系统误差。

\begin{table}[H]
\caption{$\mathcal{B}(\LtoXiXisK)$的实验测量值与理论预言值比较。}
\begin{center}
  \resizebox{\linewidth}{!}{
\begin{tabular}{l|c|c|c|c}
\hline\hline
\multirow{2}*{\minitab[c]{衰变道\\ }}                       &  \multirow{2}*{\minitab[c]{其它实验的测量 $\frac{\mathcal{B}(\Lambda^+_c\to\Xi^{(*)0}K^+)}{\mathcal{B}(\Lambda^+_c\to p K^- \pi^+)}$ \\ }}   & \multirow{2}*{\minitab[c]{PDG报道 \\ $\mathcal{B}(\Lambda^+_c\to\Xi^{(*)0}K^+)$ \\ }}  & \multirow{2}*{\minitab[c]{我们的测量结果 \\ $\mathcal{B}(\Lambda^+_c\to\Xi^{(*)0}K^+)$ \\ }}  & \multirow{2}*{\minitab[c]{理论预言 \\ $\mathcal{B}(\Lambda^+_c\to\Xi^{(*)0}K^+)$\\ }}  \\
                            &                   &                 &                &     \\ \hline
\multirow{5}{*}{$\XiK$}     &  \multirow{5}*{\minitab[c]{$(7.8\pm 1.8)\%$~\cite{Avery:1993vj} \\ }}  &    \multirow{5}*{\minitab[c]{$(5.0\pm 1.2)\times10^{-3}$~\cite{PDG2017} \\ }} &    \multirow{5}*{\minitab[c]{$(5.90\pm0.86\pm0.39)\%$ \\ }} &  2.6$\times10^{-3}$~\cite{Korner:1992wi}  \\
                            &                   &                 &                &  3.6$\times10^{-3}$~\cite{Zenczykowski:1993hw}  \\
                            &                   &                 &               &  3.1$\times10^{-3}$~\cite{Ivanov:1997ra}  \\
                            &                   &                 &                &  1.0$\times10^{-3}$~\cite{Xu:1992vc}  \\
                            &                   &                 &                &  1.3$\times10^{-3}$~\cite{Sharma:1998rd}  \\ \hline
\multirow{3}{*}{$\XisK$}    &   \multirow{3}*{\minitab[c]{$(5.3\pm1.9)\%$~\cite{Avery:1993vj}  \\ $(9.3\pm3.2)\%$~\cite{Albrecht:1994hr} }} & \multirow{3}*{\minitab[c]{$(4.0\pm 1.0)\times10^{-3}$~\cite{PDG2017} \\ }} &  \multirow{3}*{\minitab[c]{$(5.02\pm0.99\pm0.31)\times10^{-3}$ \\ }} & \multirow{3}*{\minitab[c]{5.0$\times10^{-3}$~\cite{Korner:1992wi}  \\  0.8$\times10^{-3}$~\cite{Xu:1992sw}  \\  0.6$\times10^{-3}$~\cite{Fayyazuddin:1996iy} }} \\
                            &                   &                 &                &     \\ 
                            &                   &                 &                &     \\ \hline \hline
\end{tabular}
}
\label{tab:prediction}
\end{center}
\end{table}




\section{本章小结}
本章详细介绍了BESIII实验上$\LtoXiXisK$的绝对衰变分支比测量。
使用BESIII实在$4.599\gev$处采集的567\,pb$^{-1}$ 的$\ee$对撞数据,我们看到了如图~\ref{fig:ST_datafit}所示十分清晰的$\lambdacp$信号。
利用双标记的方法模型无关地测量了$\LtoXiXisK$的绝对衰变分支比。
这一测量是$\lambdacp\lambdacm$ 对阈值上进行的首次绝对测量。
