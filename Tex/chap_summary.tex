\chapter{总结与展望}
\label{chap:summary}

\section{粲重子$\LtoXiXisK$绝对衰变分支比}

基于BESIII在 $\sqrt{s}=4.6\gev$ 处采集的567\,pb$^{-1}$的$\ee$对撞数据,运用成熟的双$\Lambda_{c}$标记技术我们可以对$\mathcal{B}(\Lambda^+_c\to \Xi^{(*)0}K^+)$进行首次的绝对测量。
测量结果可以很好的检验现有理论模型计算的是否可靠。
通过我们和最新的理论的对比发现在较好的考虑了W交换过程在粲重子衰变中的贡献之后理论计算和实验测量的一致性变得越来越好。
这再一次用实验数据说明了粲重子强衰变中W交换过程的重要性。

我们的测量精度较之前有所提高,但还有很大的改善空间。
然而我们目前的统计量还十分有限。
我们预期BESIII实验能取更大量的$\lambdacp$$\lambdacm$阈值数据。
如果BESIII合作组同意取大约$3\, \rm{fb^{-1}}$的数据,也就是目前统计量的5倍左右,我们预期粲重子$\lambdacp$主要衰变道的测量精度可以达到和粲介子相当的程度。

\section{$\psitwos$微分产生截面测量}

基于LHCb在 $\sqrt{s}=13\tev$ 处采集的275\,pb$^{-1}$~的$\pp$对撞数据,我们测量了$\psitwos$介子的微分产生截面。
采用质量-寿命联合拟合的技术,对$\pp$对撞直接产生的$\psitwos$和由$b$强子衰变而来的$\psitwos$的信号数目进行了提取。
这是在LHCb上7$\tev$质心能量下$\pp$对撞后进行的又一次$\psitwos$微分产生截面的测量。
测量采用的大统计量的真实数据保证了测量的更高精度,
测量结果可以很好的检验现有理论模型计算可靠程度。
结果中对微分产生截面与7 $\tev$下$\psitwos$发表结果的比值随$\pt$的变化进行了研究。
此外,与13$\tev$ 下$\jpsi$结果的比较也详细地研究了比值随$\pt$、$y$的变化情况。

