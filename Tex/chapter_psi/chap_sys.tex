\section{系统误差分析}
本节针对在计算微分产生截面时,由于方法的采用从而导致的系统误差进行了研究。
主要包括事例选择中的信号模型、本底模型的选择,
效率研究过程中$\mu$径迹重建效率,粒子鉴别效率、 触发效率,以及由于全事例选择、初始顶点的cut条件选择、亮度、分支比、$\pt-y$子区间边界选择、MC的统计量等引入的系统误差,还分析了极化部分导致的系统误差。以下分别进行研究。
系统误差的研究结果总结如表~\ref{tab:SystematicSummary}所示:
\begin{table}[!bp]
\caption{$\psitwos$截面测量的系统误差汇总表。其中$t_z$ 拟合只会影响到来自$b$强子衰变的$\psitwos$的截面测量。标记了$\ast$项目指的是在不同区间内该项误差是关联的。}
\centering
\begin{tabular}{l|c}
\hline
来源 & 系统误差值 (\%) \\
\hline
质量谱信号形状$^{\ast}$ & $0.0 -5.9$   \\
辐射尾巴$^{\ast}$ & $1.0$    \\
径迹重建$^{\ast}$ &  $(0.1-2.4)\oplus (2\times0.8)$ \\
粒子鉴别$^{\ast}$ &  $(0.1-0.9)\oplus(0.1-4.0)$  \\
触发$^{\ast}$   &  $0.6-7.1$  \\
$pt-y$子区间边界条件 &  $0.1 -1.8$ \\
顶点约束$^{\ast}$  & $0.4$ \\
亮度$^{\ast}$ &  $3.9$ \\
$\mathcal{B}(\psiee)^{\ast}$& $2.2$ \\
%Global event cuts$^{\ast}$   &  $0.3$ \\
\multirow{2}{*}{MC统计量} &  $0.8 -5.7$ (直接产生的\psitwos)\\
                                       &  $0.7 -12.1$ (来自$b$强子衰变的$\psitwos$)\\
$t_z$ 拟合$^{\ast}$ (只影响来自$b$强子衰变的$\psitwos$) &  $0.1 -5.9$ \\
\hline
\end{tabular}\label{tab:SystematicSummary}
\end{table}


\subsection{质量谱信号形状}
本文最初使用双CB函数描述信号,为了研究双CB模型带来的系统误差,此处采用另外一种模型(即从MC中抽取出来的模型~\cite{Cranmer:2000du})去描述信号,为了更好的描述真实数据,卷积了一个高斯函数来作为分辨函数描述信号形状。两者拟合结果的差距,作为了系统误差,误差的范围变化从0.0 到5.9\%。


\subsection{初态辐射}
由于初态辐射效应的存在,有可能会导致$\psitwos$的信号落在我们的质量窗口外边。
通过MC样本研究这种效应对截面测量的影响有$1.0\%$。

\subsection{全事例选择}
对于零级出发条件L0Dimuon而言,SPD击中的数目小于900,为了研究此cut的效率,本文采用了两个$\Gamma$ 函数来还原原本的分布,然后分别研究采用了此要求对真实数据和MC的效率差异,将两者效率的差距作为了(Global Event Cuts, GECs) 带来的系统误差。
经研究表明此项系统误差非常之小,可以安全的忽略。

\subsection{径迹重建}
$\lhcb$实验上对于径迹重建效率的研究采用的是标记-探针技术~\cite{DeCian:1402577}。
系统误差来自两方面,一是由于tracking 效率表(表如~\ref{fig:TrackEfficiencyCalib}所示)的统计误差引入的,针对此来源本文采用了toy MC的方法得到由于有限的样本统计量带来的系统误差。
这项误差取值为$0.1\%$-$2.4\%$。
事例多重数变量的选择是另一个系统误差的来源,最后安排每条径迹为0.8\% ~\cite{Aaij:2014pwa}。

\subsection{$\mu$的粒子鉴别}
针对$\mu$粒子鉴别效率的计算,误差来自于两部分。
一是有限的控制样本统计量,对于这部分系统误差我们依然采用了toy MC的方法,不同的$(\pt,y)$区间内相应的系统误差为$0.1\%$ 至 $0.9\%$;
另一个是人为的对控制样本分区间的分配模式引入的系统误差,在这里我们改变了原来的分配模式,将改变前后效率的差距 $0.1\%-4.0\%$作为了系统误差。

\subsection{触发}
为研究触发效率导致的系统误差,这里本文采用了另外一种常用方法,即TISTOS方法~\cite{LHCb-DP-2012-004}。
分别对MC和真实数据采用TISTOS 的方法,二者之间的差距做为触发效率导致的系统误差。
对于真实数据,TIS和TISTOS选择后通过拟合取其数目。
MC由于没有本底,采用数数的方式。
最终发现约$0.6\%-7.1\%$的系统误差需要考虑。

\subsection{$\pt-y$子区间边界选择}
本文采用$\pt$为1$\gevc$,$y$为0.5的间隔进行了微分产生截面的研究。
在特定的$\pt$、$y$的区间内,我们使用了该区间内的一个平均效率值来计算整个区间内的效率。
但是由于所分区间毕竟不是无限小的,在这个区间之内MC和数据在$\pt$谱和$y$谱的差异依然会引进系统误差。
我们在计算效率之前将MC的$\pt$谱和$y$谱分布重新抽样至和数据一致,然后再重新计算效率。
二者之间的差别$0.1\%-1.8\%$,作为此项系统误差。

\subsection{顶点拟合}
对于初始顶点拟合的cut条件引入的系统误差,本文通过比较真实数据和MC 样本加了这一条件的效率值之间的差距作为了系统误差,由于不同给定运动学区间内这一系统误差的浮动不是很大,所以最后取$0.4\%$作为了最终结果。

\subsection{亮度}
亮度的系统误差有专门的专家组研究,本文直接引用了该组的研究结果~\cite{Aaij:2011er}。

\subsection{分支比}
由于轻子的普适性允许本文用$\psitwos\rightarrow\e^{+}\e^{-}$代替$\psitwos\rightarrow\mu^{+}\mu^{-}$进行分支比的研究,从而引进了2.2\%的系统误差,而$\psitwos\rightarrow\mu^{+}\mu^{-}$的精度较低(10\%)。

\subsection{MC样本统计量}
有限的直接产生$\psitwos$的MC数量,带来了0.8\%-5.7\%的系统误差。
来自$b$衰变产生的$\psitwos$的MC数量,带来了0.7\%-12.1\%的系统误差。

\subsection{$t_z$拟合}
通过改变对本底参数的拟合方式,具体就是将对质量谱边带区本底的拟合改为对整个整个质量区的拟合,只不过通过sPlot技术将其中的信号提出掉。
这一项系统误差只影响来自$b$强子衰变的$\psitwos$的截面测量。其大小为$0.1\%-5.9\%$.

